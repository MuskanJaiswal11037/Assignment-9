
\usepackage[utf8]{inputenc}
\documentclass{beamer}
\usetheme{CambridgeUS}
\usepackage{listings}
\usepackage{blkarray}
\usepackage{listings}
\usepackage{subcaption}
\usepackage{url}
\usepackage{tikz}
\usepackage{tkz-euclide} % loads  TikZ and tkz-base
%\usetkzobj{all}
\usetikzlibrary{calc,math}
\usepackage{float}
\renewcommand{\vec}[1]{\mathbf{#1}}
\usepackage[export]{adjustbox}
\usepackage[utf8]{inputenc}
\usepackage{amsmath}
\usepackage{amsfonts}
\usepackage{tikz}
\usepackage{hyperref}
\usepackage{bm}
\usetikzlibrary{automata, positioning}
\providecommand{\pr}[1]{\ensuremath{\Pr\left(#1\right)}}
\providecommand{\mbf}{\mathbf}
\providecommand{\qfunc}[1]{\ensuremath{Q\left(#1\right)}}
\providecommand{\sbrak}[1]{\ensuremath{{}\left[#1\right]}}
\providecommand{\lsbrak}[1]{\ensuremath{{}\left[#1\right.}}
\providecommand{\rsbrak}[1]{\ensuremath{{}\left.#1\right]}}
\providecommand{\brak}[1]{\ensuremath{\left(#1\right)}}
\providecommand{\lbrak}[1]{\ensuremath{\left(#1\right.}}
\providecommand{\rbrak}[1]{\ensuremath{\left.#1\right)}}
\providecommand{\cbrak}[1]{\ensuremath{\left\{#1\right\}}}
\providecommand{\lcbrak}[1]{\ensuremath{\left\{#1\right.}}
\providecommand{\rcbrak}[1]{\ensuremath{\left.#1\right\}}}
\providecommand{\abs}[1]{\vert#1\vert}

\newcounter{saveenumi}
\newcommand{\seti}{\setcounter{saveenumi}{\value{enumi}}}
\newcommand{\conti}{\setcounter{enumi}{\value{saveenumi}}}
\usepackage{amsmath}
\setbeamertemplate{caption}[numbered]{}
\newcommand{\myvec}[1]{\ensuremath{\begin{pmatrix}#1\end{pmatrix}}}

\title{\typedef{ASSIGNMENT-8}}       
\author{MUSKAN JAISWAL -cs21btech11037}
\date{May 2022}
\loepackage[utf8]{inputenc}
\documentclass{beamer}
\usetheme{CambridgeUS}
\usepackage{listings}
\usepackage{blkarray}
\usepackage{listings}
\usepackage{subcaption}
\usepackage{url}
\usepackage{tikz}
\usepackage{tkz-euclide} % loads  TikZ and tkz-base
%\usetkzobj{all}
\usetikzlibrary{calc,math}
\usepackage{float}
\renewcommand{\vec}[1]{\mathbf{#1}}
\usepackage[export]{adjustbox}
\usepackage[utf8]{inputenc}
\usepackage{amsmath}
\usepackage{amsfonts}
\usepackage{tikz}
\usepackage{hyperref}
\usepackage{bm}
\usetikzlibrary{automata, positioning}
\providecommand{\pr}[1]{\ensuremath{\Pr\left(#1\right)}}
\providecommand{\mbf}{\mathbf}
\providecommand{\qfunc}[1]{\ensuremath{Q\left(#1\right)}}
\providecommand{\sbrak}[1]{\ensuremath{{}\left[#1\right]}}
\providecommand{\lsbrak}[1]{\ensuremath{{}\left[#1\right.}}
\providecommand{\rsbrak}[1]{\ensuremath{{}\left.#1\right]}}
\providecommand{\brak}[1]{\ensuremath{\left(#1\right)}}
\providecommand{\lbrak}[1]{\ensuremath{\left(#1\right.}}
\providecommand{\rbrak}[1]{\ensuremath{\left.#1\right)}}
\providecommand{\cbrak}[1]{\ensuremath{\left\{#1\right\}}}
\providecommand{\lcbrak}[1]{\ensuremath{\left\{#1\right.}}
\providecommand{\rcbrak}[1]{\ensuremath{\left.#1\right\}}}
\providecommand{\abs}[1]{\vert#1\vert}

\newcounter{saveenumi}
\newcommand{\seti}{\setcounter{saveenumi}{\value{enumi}}}
\newcommand{\conti}{\setcounter{enumi}{\value{saveenumi}}}
\usepackage{amsmath}
\setbeamertemplate{caption}[numbered]{}


\title{\typedef{ASSIGNMENT-9}}       
\author{MUSKAN JAISWAL -cs21btech11037}
\date{May 2022}
\logo{\large \Latex{}}
\begin{document}

\begin{frame}{Outline}
  \tableofcontents
\end{frame}
\section{Abstract}
	\begin{frame}{Abstract}
		\begin{itemize}
			\item 	This document contains the explanation of question  9.12  of Papoulis Pillai Probability book of chapter sequence of random variables.
		\end{itemize}
	\end{frame}
	
\maketitle

\section{PROBLEM:}
\begin{frame}{}
\begin{block}{}
Consider a general one-dimensional random walk on the possible states $e_o, e_1, e_2,$ .... 
Let $S_n$ represent the location of the particle at time n on a straight line such that at each 
interior state  $e_j$ , the particle either moves to the right to $e_{j+ 1}$ with probability $P_j$, or to 
the left to $e_{j-1}$ with probability $q_j$ or remains where it is at $e_j$.

\end{block}
\end{frame}
\section{EXPLANATION:}
\begin{frame}{}
At state $e_0$ it can either stay at the same position with probability $r_0$ or move 
to the right to $e_1$ with probability $P_1$.

The transition matrix for the given problem is:- \\
 \myvec{r_0&p_0&0&0&...\\q_1&r_1&p_1&0&...\\0&q_2&r_2&p_2&...\\0&0&q_3&r_3&p_3}\\

Random walk on a line with $ r_0+p_0=1 $ \\
$q_i+r_i+p_i=1 $ \hspace $ i=1,2,3,$......\\
Thus, $P_{00}=r_{0}$  \hspace  $p_{01}=p_0$  \hspace \hspace  $p_{0j}=0 , j>1 $ \\
and for i \ge 1 \\


$p_{ij}$=  \hspace \{  $p_i>0 $ \hspace j=i+1\\
                  $r_i >=0 $\hspace j=i\\
                  $q_i >0 $ \hspace j=i-1\\
                  0 \hspace otherwise  \}
                 
             \\
            \end{frame}
            \begin{frame}
            The model with $ p_i=p,q_i=1-p,r_i=0 $for i$>1$, and $r_0 $corresponds to the gambler's ruin problem.\\
            
            Here, the distribution of distance $d_N$ travelled after a given number of steps. Let $n_1 $ be the number 
             of steps travelled towards left and total steps be N.\\
             $d_N=2n_1-N   \\
             d=a_1 +a_2 +....a_n $\\
             $<d \ge  < \brak {a_1 +a_2 +a_3 +a_4 +.....a_n}> $\\
            $ <d\ge   <a_1>+<a_2>+<a_3>+<a_4>....<a_n>\\
              d \ge 0  $ \hspace \{As, $ \l a_i>=0$ if there is equal probability to move forward or backward and the steps taken are equal in distance.\}\\
             Average of $D^2 $\\
             $<d^2\ge <\brak{ a_1+a_2+a_3+.....a_n  }^2> $\\
            $ <d^2\ge <{a_1}^2>+<{a_2}^2>+<{a_3}^2>+.......<a_n^2> +2 \{<a_1><a_2> +<a_2><a_3>+<a_4><a_5>+.....\} $\\
            $<d^2 \ge N  $   \hspace  $\{ \sqrt{<a^2>} $  is average positive distance \}  \\
             $\sqrt{<d^2>}= \sqrt{N}$\\
             We expect  that after N steps,we are  $\sqrt{N} $ steps away from where we start. \\

            \end{frame}
\end{document}
